\documentclass{gqtekspec}
%%%%%%%%%%%%%%%%%%%%%%%%%%%%%%%%%%%%%%%%%%%%%%%%%%%%%%%%%%%%%%%%%%%%%%%%%%%%%%%%
%%
%% Filename: 	spec.tex
%%
%% Project:	wbuart32, a full featured UART with simulator
%%
%% Purpose:	To describe, for LaTeX, how to build the specification file
%%		for the wbuart32 core(s).  This file is not nearly as
%%	interesting as the file it creates, so I suggest you read spec.pdf
%%	first, before deciding you are really interested in this file.
%%
%% Creator:	Dan Gisselquist, Ph.D.
%%		Gisselquist Technology, LLC
%%
%%%%%%%%%%%%%%%%%%%%%%%%%%%%%%%%%%%%%%%%%%%%%%%%%%%%%%%%%%%%%%%%%%%%%%%%%%%%%%%%
%%
%% Copyright (C) 2015-2016, Gisselquist Technology, LLC
%%
%% This program is free software (firmware): you can redistribute it and/or
%% modify it under the terms of  the GNU General Public License as published
%% by the Free Software Foundation, either version 3 of the License, or (at
%% your option) any later version.
%%
%% This program is distributed in the hope that it will be useful, but WITHOUT
%% ANY WARRANTY; without even the implied warranty of MERCHANTIBILITY or
%% FITNESS FOR A PARTICULAR PURPOSE.  See the GNU General Public License
%% for more details.
%%
%% You should have received a copy of the GNU General Public License along
%% with this program.  (It's in the $(ROOT)/doc directory, run make with no
%% target there if the PDF file isn't present.)  If not, see
%% <http://www.gnu.org/licenses/> for a copy.
%%
%% License:	GPL, v3, as defined and found on www.gnu.org,
%%		http://www.gnu.org/licenses/gpl.html
%%
%%
%%%%%%%%%%%%%%%%%%%%%%%%%%%%%%%%%%%%%%%%%%%%%%%%%%%%%%%%%%%%%%%%%%%%%%%%%%%%%%%%
%%
%%
\usepackage{import}
\usepackage{bytefield}
\usepackage{amsmath}
\project{WBUART32}
\title{Specification}
\author{Dan Gisselquist, Ph.D.}
\email{dgisselq (at) opencores.org}
\revision{Rev.~0.1}
\begin{document}
\pagestyle{gqtekspecplain}
\titlepage
\begin{license}
Copyright (C) \theyear\today, Gisselquist Technology, LLC.

This project is free software (firmware): you can redistribute it and/or
modify it under the terms of  the GNU General Public License as published
by the Free Software Foundation, either version 3 of the License, or (at
your option) any later version.

This program is distributed in the hope that it will be useful, but WITHOUT
ANY WARRANTY; without even the implied warranty of MERCHANTIBILITY or
FITNESS FOR A PARTICULAR PURPOSE.  See the GNU General Public License
for more details.

You should have received a copy of the GNU General Public License along
with this program.  If not, see \texttt{http://www.gnu.org/licenses/} for a copy.
\end{license}
\begin{revisionhistory}
0.1 & 8/26/2016 & D. Gisselquist & Initial Draft Specification\\\hline
\end{revisionhistory}
% Revision History
% Table of Contents, named Contents
\tableofcontents
\listoffigures
\listoftables
\begin{preface}
It may be that building a UART is a mandatory coming of age task for any HDL
designer.  The task is simple, easy, and there's not all that much to it. 
This project comes out of some of my first experiences with Verilog.

Since then, it has been augmented with a very useful capability for 
simulating a UART connection when using Verilator.  It is this, perhaps
unusual, addition to the core set that makes this core worth taking note of.

I hope you find it useful.
\end{preface}

\chapter{Introduction}\label{ch:intro}
\pagenumbering{arabic}
\setcounter{page}{1}
%
% Introduction
%
% This section contains the introduction to the core, describing both its
% use and its features.
%

% What is old
The Universal Asynchronous Serial Transport, or UART, has become quite the
common protocol between devices.  It is simple to wire up, easy to use, and
easy to process.  This core provides one implementation of the logic necessary
to use such a communications scheme.

% What does the old lack?
% What is new
% What does the new have that the old lacks
% What performance gain can be expected?

While you are likely to find many UART examples out there, this particular
UART implementation offers something many of these other examples do not: a
Verilator simulation capability.  This will allow the user to connect, via
a TCP/IP port or a telnet application, to the UART of their desired chip.  As
a result, full two-way interaction can be had between a simulation and a
terminal or other port.  Indeed, this may even be sufficient to connect a 
CPU, capable of running Linux, to a terminal to verify that yes it can truly
run Linux--all within Verilator.

\chapter{Architecture}\label{ch:arch}

The HDL portion of the core itself consists of three files: {\tt rxuart.v},
{\tt txuart.v}, and {\tt wbuart-insert.v}.  These are, respectively, the
receive UART code, the transmit UART code, and an example of how the receiver
and transmitter may be connected to a Wishbone bus.

Each of the core files, {\tt rxuart.v} and {\tt txuart.v}, are fully capable.
They each accept a 29--bit setup value specifying baud rate, the number of bits
per byte (between 5 and 8), whether or not parity is used, whether that parity
is even, odd, or fixed mark or fixed space.  This setup register will be
discussed further in Chap.\ref{ch:registers}.

A further note on the {\tt rxuart.v} module is in order.  This module double
latches the input, in the proper two buffer fashion to avoid problems with
metastability.  Then, upon the detection of the start bit (i.e. a high to low
transition), the port waits a half of a baud, and then starts its baud clock
so as to sample in the middle of every baud following.  The result of this is
a timing requirement: after $N+2$ baud intervals ($N+3$ if parity is used),
where $N$ is the number of bits per byte, this calculated middle sample must
still lie within the associated bit period.  This leaves us with the criteria
that,
\begin{eqnarray}
\left|\left(N+2\right)
	\left(\frac{f_{\mbox{\tiny SYS}}}{f_{\mbox{\tiny BAUD}}}
		-{\mbox{\tt CKS}}\right)\right| 
	&<& \frac{f_{\mbox{\tiny SYS}}}{2f_{\mbox{\tiny BAUD}}},
	\label{eqn:baudlimit}.
\end{eqnarray}
where $f_{\mbox{\tiny SYS}}$ is the system clock frequency,
$f_{\mbox{\tiny BAUD}}$ is the baud rate or frequency,
{\tt CKS} is the number of clocks per baud as set in the configuration
register, and $N$ is the number of bits per byte.  What this means is that,
for transmission rates where $f_{\mbox{\tiny BAUD}}$ approaches
$f_{\mbox{\tiny SYS}}$, the number of data rates that can actually be
synthesized becomes limited.

Connecting to either {\tt txuart.v} or {\tt rxuart.v} is quite simple.  Both
files have a data port and a strobe.  To transmit, set the data and strobe
lines.  Drop the strobe line as soon as the strobe is asserted and the busy line
is not.  Likewise, to connect to the {\tt rxuart.v} port, there is a data
and a strobe.  This time, though, these two wires are outputs of the port as
opposed to inputs.
When the strobe is high, the data is valid.  It will only be high for one
clock period.  If you wish to connect this output to a bus, a register will be
needed to hold the strobe high until the data is read.  Also, while the strobe
is high, the {\tt o\_break} line will indicate whether the receiver is in a 
``break'' state, {\tt o\_frame\_err} will indicate whether or not there was a 
framing error (i.e., no stop bit), and {\tt o\_parity\_err} will indicate
wheher or not the parity matched.

The {\tt tx\_busy} line may be inverted and connected to a transmit interrupt
line.  In a similar fashion, the {\tt rx\_stb} line, or the bus equivalent of
{\tt rx\_ready}, may be used for receive interrupt lines.

An example of how to put this configuration together is found in
{\tt wbuart-insert.v}.  In this example given, the {\tt rx\_data} register
will have only the lower eight bits set if the data is valid, higher bits will
be set upon error conditions, and cleared automatically upon the next byte read.
In a similar fashion, the {\tt tx\_data} register can be written to with a byte
in order to transmit that byte.  Writing bit nine will place the transmitter
into a ``break'' condition, only cleared by writing a zero to that bit later.
Reading from the {\tt tx\_data} register can also be used to determine if the
transmitter is busy (via polling), whether it is currently in a break condition,
or even what bit is currently being placed to the output port.

The C++ simulation portion of the code revolves around the file
{\tt bench/cpp/uartsim.cpp} and its associated header.  This file defines a 
class, {\tt UARTSIM}, which can be used to connect the UART to a TCP/IP stream.
When initialized, this class takes, as input, the TCP/IP port number that the
class is to connect with.  Once connected, using this is as simple as
calculating the receive input bit from the transmit output bit when the
clock is low, and the core takes care of everything else.

\chapter{Operation}\label{ch:ops}

% This section describes the operation of the core.  Specific sequences, such
% as startup sequences, as well as the modes and states of the block should be
% described.
%

To use the core, a couple of steps are required.  First, wire it up.  The
{\tt wbuart-insert.v} file should provide a good example of how to wire it up.
Second, set the UART configuration register.  This is ideally set in an
initial statement within the code somewhere, but can easily be set elsewhere
by writing to this register from the bus.

From a simulation standpoint, it will also need to be wired up.  Somewhere, 
internal to the top--level Verilator C++ simulation file, you'll want to
have a line similar to,
\begin{tabbing}
{\tt if (!clk)} \= \\
\> {\tt tb->i\_rx} {\tt = } {\tt uartsim(tb->o\_uart, setup);}
\end{tabbing}

To use the transmitter, set the {\tt i\_stb} and {\tt i\_data} wires.  Drop
the strobe line any time after {\tt (i\_stb)\&\&(!o\_busy)}.

To use the receiver, grab the data any time {\tt o\_stb} is true.

From the standpoint of the bus, there are two ways to handle receiving and
transmitting: polling and interrupt based, although both work one character at
a time.  To poll, repeatedly read the receive data register until only no
bits but the bottom eight are set.  This is an indication that the byte is
valid.  Alternatively, you could wait until the an interrupt line is set and
then read.  In the {\tt wbuart-insert.v} example, the {\tt rx\_int} line will
be set, and automatically cleared upon any read.  To write, one can read from
the transmit data register until the eighth bit, the {\tt tx\_busy} bit, is
cleared, and then transmit.  Alternatively, this negation of this bit may be
connected to an interrupt line.  Writing to the port while idle will start
it transmitting.  Writing to the port while it is busy will fill a one word
buffer that will get sent as soon as the port is idle for one clock.


\chapter{Registers}\label{ch:registers}
% This section specifies all internal registers.  It should completely cover
% the interface between the CPU and the host as seen from the software point
% of view.

% List of Registers

% Register 1 Description
%
% You shall choose the style of register you prefer.  Do not use both options
% in one and the same document.  (Table of bits, vs. byetarray type of
% description).

The core really only has one register associated with it, which is the setup
register.  The format of this register is important, although not necessarily
trivial or obvious.  We'll cover two other registers here, though, associated
with the example wishbone connections from {\tt wbuart-insert.v}.  All three
of these registers are shown in Tbl.~\ref{tbl:reglist}.
\begin{table}
\begin{center}
\begin{reglist}
SETUP	& & 30 & R/W & UART configuration/setup register.\\\hline
RX\_DATA & & 12 & R(/W) & Read data, reads from the UART.\\\hline
TX\_DATA & & 12 & (R/)W & Transmit data: writes send out the UART.\\\hline
\end{reglist}\caption{UART Registers}\label{tbl:reglist}
\end{center}\end{table}

Since the connections presented are only examples, they are listed without
addresses, as their wishbone bus connectivity will be determined once they 
are connected.

\section{Setup Register}
The setup register is perhaps the most critical of all the registers.  This
is shown in Fig.\ref{fig:SETUP}.
\begin{figure}\begin{center}
\begin{bytefield}[endianness=big]{32}
\bitheader{0-31}\\
\bitbox{2}{00}
\bitbox{2}{N}
\bitbox{1}{S}
\bitbox{1}{P}
\bitbox{1}{F}
\bitbox{1}{T}
\bitbox{24}{Baud CLKS}
\end{bytefield}
\caption{SETUP Register fields}\label{fig:SETUP}
\end{center}\end{figure}
It is designed so that, for any 8N1 protocol (eight data bits, no parity, one
stop bit), only the number of clocks per baud interval needs to be set.  The
top two bits are unused, making this a 30--bit number.  The other fields
are: $N$ sets the number of bits per word.  A value of zero corresponds
to 8--bit words, a value of one to seven bit words, and so forth up to a value
of three for five bit words.  $S$ determines the number of stop bits.  Set this
to one for two stop bits, or leave it clear for a single stop bit.  $P$
determines whether or not a parity bit exists (1 for parity, 0 for none),
while $F$ determines whether or not the parity is fixed.  Tbl.~\ref{fig:parity}
lists out the various values possible here.
\begin{table}\begin{center}
\begin{tabular}{ccc|l}
P&F&T&Setting \\\hline\hline
1 & 0 & 0 & Odd parity		\\\hline
1 & 0 & 1 & Even parity	\\\hline
1 & 1 & 0 & Parity bit is a Space (1'b0)\\\hline
1 & 1 & 1 & Parity bit is a Mark (1'b1)\\\hline
0 & & & No parity \\\hline
\end{tabular}\caption{Parity setup}\label{tbl:parity}
\end{center}\end{table}

\section{RX\_DATA Register}
Fig.~\ref{fig:RXDATA}
\begin{figure}\begin{center}
\begin{bytefield}[endianness=big]{32}
\bitheader{0-31}\\
\bitbox{20}{20'h00}
\bitbox{1}{B}
\bitbox{1}{F}
\bitbox{1}{P}
\bitbox{1}{S}
\bitbox{8}{RWORD}
\end{bytefield}
\caption{RXDATA Register fields}\label{fig:RXDATA}
\end{center}\end{figure}
breaks out the various bit fields of the receive
data register used in the {\tt wbuart-insert.v} example of connecting it to 
a bus.  In particular, the $B$ field indicates that the receive line is in 
a break condition.  The $F$ and $P$ fields indicate that a frame error or
parity error were detected.  These are valid like the data word: when the strobe
line is set.  The $S$ field will be false when the {\tt RWORD} is valid.
Hence, if {\tt (RWORD \& ~0x0ff)} is zero there is a word ready to be received
without error.

\section{TX\_DATA Register}
Fig.~\ref{fig:TXDATA}
\begin{figure}\begin{center}
\begin{bytefield}[endianness=big]{32}
\bitheader{0-31}\\
\bitbox{20}{2'h00}
\bitbox{1}{C}
\bitbox{1}{O}
\bitbox{1}{B}
\bitbox{1}{S}
\bitbox{8}{TWORD}
\end{bytefield}
\caption{TXDATA Register fields}\label{fig:TXDATA}
\end{center}\end{figure}
breaks out the various bit fields of the transmit data register used in
{\tt wbuart-insert.v}.  The $C$ field indicates whether or not the receive
data line is high or low, the $O$ field indicates the same for the transmit
line.  These aren't particularly useful or valuable, but they don't fit in the
receive data register since they would violate the error condition detector.
They're thrown in here for whatever useful purpose one might find.  The $B$
field, when set, sends a break condition down the wire.  Writing to the TXDATA
register, clearing the $B$ field, will clear the transmitter from the break 
condition without transmitting anything.  The $S$ field is similar to the RXDATA
strobe register.  It will be true whenever the transmitter is busy or a byte
is waiting for it.  It will be clear only when the transmitter is idle.

To use the transmitter, simply write a byte to the TXDATA register with the
upper 24--bits clear to transmit.

\chapter{Clocks}\label{ch:clocks}
The UART has been tested with a clock as fast as 200~MHz
(Tbl.~\ref{tbl:clocks}). 
\begin{table}\begin{center}
\begin{clocklist}
{\tt i\_clk} & (System) & 200~MHz & & System clock\\\hline
\end{clocklist}
\caption{Clock Requirements}\label{tbl:clocks}
\end{center}\end{table}
It should be able to use slower clocks, but only subject to the ability to
properly set the baud rate as shown in Eqn.~\eqref{eqn:baudlimit} on
Page~\pageref{eqn:baudlimit}.

I do not recommend using this core with a baud rate greater than a quarter
of the system clock rate.

% This section specifies all of the clocks.  All clocks, clock domain passes
% and the clock relations should be described.

% Name | Source |    Rates (MHz)    | Remarks | Description
%               | Max|Min|Resolution|
\chapter{Wishbone Datasheet}\label{ch:wishbone}

Tbl.~\ref{tbl:wishbone}
\begin{table}[htbp]
\begin{center}
\begin{wishboneds}
Revision level of wishbone & WB B4 spec \\\hline
Type of interface & Slave, Read/Write, pipeline reads supported \\\hline
Port size & 32--bit \\\hline
Port granularity & 32--bit \\\hline
Maximum Operand Size & 32--bit \\\hline
Data transfer ordering & (Irrelevant) \\\hline
Clock constraints & None.\\\hline
Signal Names & \begin{tabular}{ll}
		Signal Name & Wishbone Equivalent \\\hline
		{\tt i\_wb\_clk} & {\tt CLK\_I} \\
		{\tt i\_wb\_cyc} & {\tt CYC\_I} \\
		{\tt i\_wb\_stb} & {\tt STB\_I} \\
		{\tt i\_wb\_we} & {\tt WE\_I} \\
		{\tt i\_wb\_addr} & {\tt ADR\_I} \\
		{\tt i\_wb\_data} & {\tt DAT\_I} \\
		{\tt o\_wb\_ack} & {\tt ACK\_O} \\
		{\tt o\_wb\_stall} & {\tt STALL\_O} \\
		{\tt o\_wb\_data} & {\tt DAT\_O}
		\end{tabular}\\\hline
\end{wishboneds}
\caption{Wishbone Datasheet}\label{tbl:wishbone}
\end{center}\end{table}
is required by the wishbone specification in order to declare the core as
wishbone compliant, and so it is included here.  It references the connections
exemplified by {\tt wbuart-insert.v}.  The big thing to notice is that this core
acts as a wishbone slave, and that all accesses to the core
registers are 32--bit reads and writes to this interface.

What this table doesn't show is that all accesses to the port take a single
clock.  That is, if the {\tt i\_wb\_stb} line is high on one clock, the
{\tt i\_wb\_ack} line will be high the next.  Further, the {\tt o\_wb\_stall}
line is tied to zero. 

Also, this particular wishbone implementation assumes that if {\tt i\_wb\_stb},
then {\tt i\_wb\_cyc} will be high as well.  Hence it only checks whether or not
{\tt i\_wb\_stb} is true to determine if a transaction has taken place.  If your
bus does not meet this requirement, you'll need to AND {\tt i\_wb\_stb} with
{\tt i\_wb\_cyc} before using the core.

\chapter{I/O Ports}\label{ch:ioports}
% This section specifies all of the core IO ports

In it's simplest form, the UART offers simply two I/O ports: the {\tt i\_rx}
line to receive, and the {\tt o\_tx} line to transmit.  These lines need to be
brought to the outside of your design.  Within verilator, they need to be
connected inside your verilator test bench, as in:
\begin{tabbing}
{\tt if (!clk)} \= \\
\> {\tt tb->i\_rx} {\tt = } {\tt uartsim(tb->o\_uart, setup);}
\end{tabbing}

A more detailed discussion of the connections associated with these modules
can begin with Tbl.~\ref{tbl:rxports}, detailing the I/O ports of the
UART receiver, and Tbl.~\ref{tbl:txports},
\begin{table}\begin{center}\begin{portlist}
{\tt i\_clk}	& 1 & Input & The system clock \\\hline
{\tt i\_reset}	& 1 & Input & A positive, synchronous reset \\\hline
{\tt i\_setup}	& 30 & Input & The 30--bit setup register \\\hline
{\tt i\_uart}	& 1 & Input & The input wire from the outside world. \\\hline
{\tt o\_wr}	& 1 & Output & True if a word was received.  At this time,
		{\tt o\_data}, {\tt o\_break}, {\tt o\_parity\_err}, and
		{\tt o\_frame\_err} will also be valid. \\\hline
{\tt o\_data}	& 8 & Output & The received data, valid if {\tt o\_wr} \\\hline
{\tt o\_break}	& 1 & Output & True in the case of a break condition \\\hline
{\tt o\_parity\_err}	& 1 & Output & True if a parity error was detected \\\hline
{\tt o\_frame\_err}	& 1 & Output & True if a frame error was detected \\\hline
{\tt o\_ck\_uart}	& 1 & Output & A synchronized copy of {\tt i\_uart} \\\hline
\end{portlist}\caption{RXUART port list}\label{tbl:rxports}
\end{center}\end{table}
detailing the I/O ports of the UART transmitter.
\begin{table}\begin{center}\begin{portlist}
{\tt i\_clk}	& 1 & Input & The system clock \\\hline
{\tt i\_reset}	& 1 & Input & A positive, synchronous reset \\\hline
{\tt i\_setup}	& 30 & Input & The 30--bit setup register \\\hline
{\tt i\_break}	& 1 & Input & Set to true to place the transmit channel into a break condition\\\hline
{\tt i\_wr}	& 1 & Input & An input strobe.  Set to one when you wish to transmit data, clear once it has been accepted\\\hline
{\tt i\_data}	& 8 & Input & The data to be transmitted, ignored unless
		{\tt (i\_wr)\&\&(!o\_busy)} \\\hline
{\tt o\_uart}	& 1 & Output & The wire to be connected to the external port\\\hline
{\tt o\_busy}	& 1 & Output & True if the transmitter is busy, false if it will receive data\\\hline
\end{portlist}\caption{TXUART port list}\label{tbl:txports}
\end{center}\end{table}

The ``ports'' associated with the {\tt wbuart-insert.v} example may be
inferred from the wishbone data sheet.

% Appendices
% A. May be added to outline different specifications.  (??)


% Index
\end{document}


